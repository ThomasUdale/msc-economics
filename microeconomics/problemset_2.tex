\documentclass{article}
\usepackage{amsmath}
\usepackage{derivative}

\begin{document}
    \begin{enumerate}

        \item The utility function is $u(x_1,x_2) = x_1x_2$
        \begin{enumerate}
            \item Using the definition of indirect utility and duality:
            \begin{align*}
                v(p_1,p_2,m) = \frac{m^2}{4p_1p_2} \\
                v(p_1,p_2,e(p_1,p_2,u)) = u \\
                e(p_1,p_2,u)^2 = 4p_1p_2u \\
                e(p_1,p_2,u) = \sqrt{4p_1p_2u}
            \end{align*}
            \item CV and EV:
            \begin{enumerate}
                \item Mr Watt's would take a £50 pay cut to achieve the same utility as he currently has.
                \begin{align*}
                    e(p_1,v(p_0,m_0)) - m_0 = \sqrt{4 * \frac{1}{4} * 1 * \frac{100^2}{4 * 1 * 1}} - 100 = -50
                \end{align*}
                \item A £100 increase pay would be required to achieve the same level of utility as in the new town.
                \begin{align*}
                    m_1 - e(p_0,v(p_1,m_1)) = 100 - \sqrt{4 * 1 * 1 * \frac{100^2}{4 * \frac{1}{4} * 1}} = -100
                \end{align*}
            \end{enumerate}
            \item The consumer surplus will be between -50 and -100.
        \end{enumerate}
        
        \item 
        \begin{enumerate}
            \item Hotelling's Lemma says that if $y^*,x^*$ are the solution to the PMP for a firm then we have that:
            \begin{align*}
                \pdv{\pi}{p} = y^*(p,\textbf{w}) \\
                \pdv{\pi}{w_i} = -x^*_i(p,\textbf{w})
            \end{align*}
            To see this is true, consider the PMP:
            \begin{align*}
                \max_{y,x} py(p) - \textbf{wx}
            \end{align*}

            and apply Shepard's Lemma to the optimal value function $\pi(p, \textbf{w})$.

            \item If $w_i$ increases, then we can see that profit must decrease, as $x_i > 0$.
            \item \begin{align*}
                \max_{L} pL^\frac{1}{2} - wL \\
                \frac{1}{2}pL^{-\frac{1}{2}} - w = 0 \\
                L = (\frac{p}{2w})^2 \\
                \pi = \frac{p^2}{2w} - \frac{p^2}{4w} \\
                \pdv{\pi}{w} = -\frac{p^2}{2w^2} + \frac{p^2}{4w^2} = -\frac{p^2}{4w^2} = -L
            \end{align*}
        \end{enumerate}

        \item \begin{enumerate}
            \item \begin{align*}
                40p - 2pz - w = 0
            \end{align*}
            The second derivative is $-2$, therefore this solution is profit maximising. 
            \item 
            \begin{enumerate}
                \item If $z^* = 0$, then $w = 40p$.
                \item If $z^* = 20$, then $w = 0$ and $p \neq 0$.  
            \end{enumerate}
            \item The input demand function is $z^* = \frac{40p - w}{2p} = 20 - \frac{w}{2p}$. The output supply function is:
            \begin{align*}
                y^* &= 40z^* - {z^*}^2 \\
                &= 40(20 - \frac{w}{2p}) - (20 - \frac{w}{2p})^2 \\
                &= 800 - \frac{20w}{p} - 400 + \frac{20w}{p} - \frac{w^2}{4p^2}\\
                &= 400 - \frac{w^2}{4p^2}\\
                \pi &= py(p) - wz \\
                \pi &= 400p - \frac{w^2}{4p} - 20w + \frac{w^2}{2p} 
            \end{align*}
            \item Hotelling's Lemma:
            \begin{align*}
                \pdv{\pi}{w} &= -(20 - \frac{w}{2p})\\
                &=-z^*(p,w)
            \end{align*} 
        \end{enumerate}
    \end{enumerate}
\end{document}